%%%   Important note: think of a good write of the educational plan, that
%%is our plans fro training. Now it says nothing
\subsection{Proposed Research}


In this section, we describe the detailed components of our proposed research that was 
outlined in the Overview section. In the following project descriptions, Bogner will lead the 
IMSRG-related developments, while Hjorth-Jensen will lead the coupled-cluster theory developments.



\subsubsection{{\em Ab initio} calculations of Finite Nuclei}\label{subsec:finitenuclei}

The projects in this subsection concern the development of {\em ab
  initio} methods to calculate ground- and excited-state properties of
medium mass nuclei. The methods we will focus on are the IMSRG and CC theory. Since we
have already developed a large MBPT and large-scale FCI framework \cite{Hjorth-Jensen:1995ys,mhj2004,Tsunoda:2013bla,Tsunoda:2016fjh}, these methods will serve as
additional many-body methods used to compare our calculations with.
Emphasis will be placed on developing reliable and efficient {\em ab initio}
calculations of ground- and excited-state properties for both closed-
and open-shell nuclei, with the proper inclusion of two- and
three-nucleon forces and continuum degrees of freedom.  Additionally,
these developments will be used to inform and improve existing
phenomenological approaches, for instance by generating improved
starting points for the optimization of phenomenological shell model
Hamiltonians and energy density functionals, or by providing
qualitative insights as to the different types of couplings one
expects.



\begin{enumerate}
 
 \item {\bf Inclusion of NNN forces and three-body correlations.} 
 In the same way that NN interactions imply that, at a minimum, doubles excitations ($2p2h$,
  disconnected $4p4h$, etc.) must be included for an accurate
  description of many-body correlations, the presence of NNN forces
  demands an explicit treatment of triples excitations. We propose to 
  develop computationally feasible schemes for including explicit
  three-body correlations in CC and IMSRG calculations when NNN forces are present.  In CC theory, this corresponds to including
  non-iterative triples (e.g., CCSD(T)) and analogous approximations to full triples, while for the IMSRG, this entails performing approximate IMSRG(3)
  calculations using the Magnus expansion formulation developed in Refs.~\cite{TitusThesis, Morris:2015ve}. 
s
\item{\bf Merging EOM methods with the single-reference IMSRG.}
In Ref.~\cite{Parzuchowski:2016njm}, we presented a new approach to 
 calculate excited states in medium-mass closed-shell nuclei by merging 
 equations-of-motion (EOM) techniques~\cite{Rowe:1968eq} with the IMSRG.  We propose to extend this method to calculate ground- and excited-state properties of nuclei within 1-2 nucleons of a closed shell by extending the EOM ladder operators to include particle-number non-conserving terms. We will also use Lanczos' method and appropriately evolved operators to calculate nuclear response functions and transition densities. In principle, the same Lanczos techniques can be used to construct microscopic optical potentials when applied to the closed-shell $\pm 1$ nuclei, thereby linking \emph{ab initio} structure to reactions.
 
\item {\bf Merging EOM methods with the multi-reference IMSRG.}  The
  valence-space IMSRG~\cite{Tsukiyama:2012fk,Bogner:2014tg,
    Stroberg:2015ymf,Stroberg:2016ung} has the virtue of providing a
  unified treatment of ground- and excited-state properties couched in the familiar language of the shell model, but it also suffers the same ``curse
  of dimensionality'' associated with the large-scale matrix
  diagonalizations required for midshell nuclei and/or
  extended valence spaces.  One alternative is to directly target excited states by combining the
  IMSRG with equations-of-motion (EOM) techniques~\cite{Rowe:1968eq},
  similar to what is done in CC
  theory~\cite{Ekstrom:2014iya,Jansen:2012ey}. Successful
  ``proof-of-principle'' calculations have been performed in
  Ref.~\cite{Parzuchowski:2016njm}, where EOM methods were combined
  with the single-reference IMSRG to calculate excited states in
  closed-shell systems. While the EOM-IMSRG potentially offers some
  technical simplifications due to the Hermiticity of the transformed
  Hamiltonian (e.g., no need to solve a separate left-eigenvalue
  problem when calculating properties other than energy), the
  practical limitations of the single-reference formulation should be
  comparable to the analogous EOM-CC calculations, limiting the method
  to nuclei within 1 or 2 nucleons of a closed shell. To remove these
  limitations, we propose to merge EOM techniques with the
  multi-reference IMSRG (MR-IMSRG) formulation recently developed for
  ground-state calculations of open-shell even-even
  nuclei~\cite{Hergert:2013ij,Hergert:2014vn}. In principle,
  spectroscopy for the target nucleus and its even-odd, odd-even, and
  odd-odd neighbors could then be accessed using suitably generalized
  EOM excitation operators. This work will be done in close collaboration with
  NSCL colleague Heiko Hergert.
  
\item {\bf Continuum effect via the Gamow basis.}  As one nears the
  limits of stability, \emph{ab initio} calculations in a localized,
  $L^2$ single-particle basis fail to account for important couplings
  to resonances and the non-resonant continuum. In recent years, we
  have had success in working with the so-called Gamow basis in MBPT
  and coupled cluster calculations for nuclei near shell closures, see
  Ref.~\cite{Hagen:2006pq,Tsukiyama:2009hy,Hagen:2012oq} and
  references therein. To date, none of the IMSRG implementations have
  incorporated this important physics. Therefore, a major component of
  this proposal will be to implement the Gamow basis in the EOM-IMSRG,
  valence-shell decoupling IMSRG, and MR-IMSRG formulations, with
  particular emphasis placed on the latter two since they span the
  more challenging regime of unstable, open-shell nuclei. 
%  Our multi-shell MBPT formalism wil be extended in order to 
%include continuum effects as well as three-body forces, providing thereby three different many-body approaches to the inclusion of continuum effects.
  
\item {\bf Shell model interactions in extended valence spaces.} 
As one nears the limits of stability, traditional shell model calculations in a
single major shell are often insufficient. Moreover, for loosely-bound
systems it is necessary to utilize a single-particle basis which
accounts for resonances and the non-resonant continuum. Therefore, it
is a high priority to develop a framework to derive
multi-shell effective interactions with the proper inclusion of
continuum degrees of freedom using both CC and IMSRG methods. We note that we have recently tested multi-shell
effective interactions derived in MBPT for selected nuclei, see
Ref.~\cite{Tsunoda:2013bla,Tsunoda:2016fjh}. We propose to extend the CC effective interaction (CCEI)~\cite{Jansen:2014qf,Jansen:2015ngw} and the valence-decoupled
IMSRG methods~\cite{Bogner:2014tg,Stroberg:2015ymf,Stroberg:2016ung}, both of which have been quite successful in calculations within a single major shell and neglecting continuum effects, to the more
challenging case of multiple major shells within the Gamow basis.  Together with the previous item and with our
multi-shell effective interactions from MBPT, we have the possibility to test three different many-body approaches, with and without three-body forces and continuum effects.


\item{\bf Microscopically-improved phenomenological shell model
    interactions.}  In the phenomenological approach to shell model
  interactions, the valence single-particle energies (SPEs) and
  two-body matrix elements (TBMEs) are optimized directly to data. In
  practice, this is done by starting with some microscopic
  calculation, and then iteratively adjusting the relevant linear
  combinations of SPEs and TBMEs that are probed by the data, leaving
  the remaining terms fixed at the values given by the microscopic
  calculation~\cite{Brown:1988vm,Brown:2001rg,Caurier:2005qf}. In all
  applications to date, the microscopic starting point is taken from a
  low-order MBPT calculation without initial NNN forces. We propose to
  use our multi-shell MBPT, CCEI and IMSRG shell model interactions, which start from
  state-of-the-art chiral NN and NNN interactions, as an improved starting point for the
  optimization of phenomenological interactions in collaboration with NSCL colleague Alex Brown. We will also use
  qualitative and semi-analytic insights gleaned from the CCEI and
  IMSRG calculations to suggest alternative ans\"atze for the
  phenomenological interactions. For instance, in traditional
  phenomenology, the TBMEs are essentially fixed across the entire
  shell, with the exception of an empirical $A$-dependent scaling that
  mocks up the changing size of the
  core~\cite{Brown:1988vm,Brown:2001rg}. In contrast, the IMSRG
  valence interactions seem to suggest a much more complicated
  $A$-dependence~\cite{Stroberg:2015ymf,Stroberg:2016ung}.  We will
  also explore how the initial Hamiltonian is renormalized under the
  IMSRG to gain some insight into the general operator structures that
  result, which can then be used to motivate a ``bottom-up''
  description of the shell model as an effective field theory around
  the Fermi surface, in analogy with the modern RG view of
  Landau-Migdal theory~\cite{ShankarRG,MigdalBook}.
    
\end{enumerate}



\subsubsection{{\em Ab initio} calculations of infinite matter}\label{subsec:infinitematter}

Starting from optimized chiral EFT Hamiltonians we will perform extensive
studies of the equation of state of nucleonic matter, extracting
constraints on the energy per particle in symmetric matter and
$\beta$-stable matter, and obtaining reliable estimates for the
symmetry energy.  Our research aims at providing a consistent approach to several crucial quantities of relevance
for dense matter studies, spanning from the equations of state and the symmetry energy 
to various response functions and the estimation of  neutrino rates, spectra and opacities. These are all
quantities which play prominent roles in  our understanding of the evolution of neutron stars and proto-neutron stars.
The role of three-body forces on these quantities will be explored as well using realistic chiral Hamiltonians.

\begin{enumerate}
\item{\bf EoS and symmetry energy calculations.} We will develop IMSRG and CC calculations of
  the EoS and the symmetry energy of infinite matter starting from
  realistic NN and NNN interactions with approximate and full triples excitations. In our contributions to a recent
  volume of Lecture Notes in Physics (to be published in early
  2017), entitled {\em An advanced course in computational nuclear
   physics; Bridging the scales from quarks to neutron stars}
  \cite{lnp} (see also Refs.~\cite{Baardsen:2013vwa,Hagen:2013yba}), we have
  developed IMSRG and CC infinite matter codes at the IMSRG(2) and CCD levels of approximation. Building on this foundation,
  we will develop approximate IMSRG(3) and CCDT codes that are capable of treating realistic chiral NN and NNN interactions. These calculations will provide reliable   benchmarks for the EoS of general asymmetric nuclear matter and
$\beta$-stable nuclear matter. We have already implemented approximate 
  triples correlations in studies of the homogeneous electron gas for the coupled-cluster
  \cite{audunthesis} and IMSRG~\cite{TitusThesis} methods. Due to large memory requirements, our codes will need to be rewritten in order to run efficiently on existing and
  future high-performance facilities.  Finally, in addition to providing valuable information about the symmetry
  energy and its dependence on various components
  of the underlying NN and NNN interactions, these
  calculations will allow us to extract the so-called Landau fermi
  liquid parameters, which are useful to parameterize the collision
  integral term in transport theory.  This will naturally complement
  the reaction theory program of our colleagues Danielewicz and Nunes at the
  NSCL at Michigan State University.

% Should ask Luke to take a look at this? 

\item{\bf Effective operators and Neutrinos properties.} The CC and
  IMSRG methods will also allow us to extract one- and two-body
  effective operators for asymmetric infinite matter and to study
  their dependence on the various components of chiral NN and NNN
  interactions.  As an example, we will develop formalism and codes
  for computing effective baryon masses in dense matter as well as
  effective medium-dependent interactions and operators. These
  ingredients form the basis for the evaluation of various processes
  for neutrino emissions in dense matter. In neutron stars, neutrinos
  exhibit a mean free path which is larger than the typical radius
  $R\sim 10$ km of a neutron star. Neutron stars are thus expected to
  cool via various neutrino emission mechanisms such as the direct and
  modified Urca processes, and neutrino-pair bremsstrahlung. To
  properly account for the medium dependence of these processes, one
  needs controlled calculations of quantities such as nucleon
  effective masses, effective operators and interactions, and
  spin-isospin response functions. These elements all enter the
  determination of the rates that characterize the different neutrino
  production mechanisms. Normally, these have been approximated
  through various mean-field approaches or simplified phenomenological
  interactions. Changes in such rates due to uncertainties in
  different many-body methods can significantly alter the
  neutrino cooling timescale of proto-neutron stars and impact
  predictions of nucleosynthesis in the innermost regions of core
  collapse supernovae.  For instance, a proper knowledge of the
  spin-isospin response from {\em ab initio} methods, even in a
  limited range of temperature-density space, would provide valuable
  constraints on the phenomenological models that are in current
  use. The aim of this part of the project is thus to compute neutrino
  emissivities and spectra in dense matter using medium dependent
  quantities like effective masses, effective one-body operators and
  two-body interactions, and spin-isospin response functions extracted from IMSRG and CC calculations with
  and without NN and NNN forces. The hope is then to be able to
  provide better constrained predictions for neutrino spectra in dense
  matter.

  
\item{\bf Finite size corrections and many-body approximation errors.}
 In this subproject, we will treat the results from our large-scale simulations as ``synthetic data''  that depend on the number of basis functions used and the truncations
we make at the level of many-body excitations. By analyzing trends and correlations within the data, we will extract empirical guidelines and ``rules of thumb'' to estimate the theoretical uncertainties of common many-body approximations. 

  Since we simulate infinite matter in a finite periodic box and truncated plane-wave basis for both the
  IMSRG and CC methods, an important aspect will be to study and
  quantify finite-size and basis-truncation errors. For the former, we propose to study the use of
  so-called twisted periodic boundary conditions, which have been used
  with much success in condensed matter calculations to minimize
  finite size effects~\cite{Shepherd:2012,Hagen:2013yba,Drummond:2008}.  In
  depth studies of a proper treatment of finite size effects will be
  included in both our IMSRG and CC codes. Our numerical results for a wide range of basis sets will allow us to extract
  a better understanding of the basis-set truncation errors. An analysis along the lines performed
  by Shepherd in \cite{shepherd2016} will be done using our numerical data.
   
  We will also investigate truncation errors arising from different
  levels of many-body approximation. Unlike the studies of basis-set
  truncation errors, it is difficult to obtain an analytic or
  semi-analytic understanding of many-body approximation errors for
  nuclear many-body systems. Therefore, we will proceed in an
  empirical fashion by using benchmark results provided from auxiliary field diffusion Monte
  Carlo calculations~\cite{Roggero:2013ora,Roggero:2014lga,pederiva2016} to generate quasi-exact 
  results for a range of nuclear densities.  Using the
  Monte Carlo results as the exact benchmark, we will study how the
  truncation errors of the IMSRG, CC, and MBPT in different levels of approximation
  correlate with various quantities that one expects to affect convergence, such as the UV cutoff in the nuclear
  Hamiltonian, the D-state probability in the deuteron, the wound
  parameter calculated in leading-order Brueckner Hartree-Fock theory,
  etc. It is hoped that in this manner, we can at least gain ``rule of
  thumb'' estimates of how accurate we expect a given level of CC or
  IMSRG calculation to be given some simple quantities that
  characterize the input interactions and the reference state upon
  which correlations are to be built.   The calculations will be performed with and without three-body forces as well as with simplified Hamiltonians like those provided from pionless effective field theory \cite{pederiva2016}. The latter references, based on 
auxiliary field diffusion Monte Carlo calculations provide near exact benchmarks to neutron matter and symmetric nuclear matter calculations. 
 \end{enumerate}


\subsubsection{Applications to other systems}\label{subsec:othersystems}
To further validate our many-body machinery, we will also carry out
ground- and excited-state calculations using CC, IMSRG and MBPT
for the following well-studied systems that can serve as interesting
benchmarks for our methods:
\begin{enumerate}
\item{\bf Homogenous electron gas (HEG) in two and three dimensions.}
  The HEG has been studied extensively in the literature, see for
  example the recent full configuration interaction theory quantum Monte Carlo (FCIQMC) calculations of Alavi {\em et al}
  \cite{booth2009,booth2013}. We have recently started to study the
  two-dimensional HEG using CC theory and IMSRG,
  with very promising results~\cite{Baardsen:2013vwa,audunthesis}. These systems can
  serve as a testbed for the development of reliable infinite nuclear
  matter calculations, as much of the formalism is similar except that
  the nuclear forces have a finite range compared with the Coulomb
  interaction. Since Coulomb interactions are simpler, studies of the
  HEG will allow us to gain important insights about many-body
  correlations and various approximations like CC theory with and
  without triples correlations, IMSRG(2) and IMSRG(3) etc. In our
  IMSRG, CC and MBPT calculations to date, periodic boundary
  conditions have been used without including finite-size
  corrections~\cite{Baardsen:2013vwa, audunthesis}. Here again we plan to perform calculations with full triples
 and perform similar benchmark calculations as done for our nuclear matter calculations. 
  In order to compare the results to
  the benchmarks from Monte Carlo calculations, we will have to implement finite-size corrections in our
  formalism. Since we ultimately face the same issues in the nuclear
  matter problem, we believe that the conclusions and experiences we
  will gain from the HEG will be very useful.  The analysis of the approximation errors discussed in the previous subproject, will be performed here as well.
 
\item{\bf Quantum dots and neutron drops.} We have also performed
 test studies of quantum dots in two-dimensional parabolic
  traps~\cite{Lohne:2010aw,sarahthesis}.  We have benchmarked CCSD and CCSDT theories and IMSRG(2) with
  diffusion Monte Carlo calculations, obtaining results which agree
  rather well. The aim is to extend the studies of such systems
  open-shell configurations and double well quantum dots by implementing the equation of motion techniques discussed above. 
  We have already performed preliminary calculations of the spectra of systems with one electron added or removed from a closed shell core, using CC and IMSRG at the level of doubles excitations. A comparison of MBPT using the IMSRG-evolved interaction to third order shows good agreement for addition and removal spectra. These results will be published early 2017 with three of our graduate students. We note that the formalism developed here can in turn be used to extract effective masses and effective single-particle energies for nucleons in dense matter, electrons in the HEG and for closed-shell nuclei with one particle removed or attached.
 Finally, as the problem is essentially identical to quantum dots, we will
  also perform calculations of harmonically-trapped neutrons
  (``neutron drops''), which serve as useful theoretical ``data'' to
  constrain the construction of improved energy density functionals  for
  nuclei~\cite{Bogner:2011kp}.

\item{\bf Unitary fermi gas.}  Another interesting testing ground for
  many-body theories is the unitary fermi gas~\cite{UFG}. The unitary
  fermi gas (UFG) is a many-fermion system interacting via short-range
  interactions tuned to infinite $s$-wave scattering length. Since the
  only length scale in the problem is the Fermi momentum, many
  properties of the UFG obey elegant universal relations that are
  independent of the details of the microscopic interactions. As a
  result, the physics of the UFG is relevant to a wide range of
  systems from trapped atomic gases to low-density neutron matter to
  the quark-gluon plasma~\cite{UFG}.  Despite the preponderance of
  elegant and simple universal relations, the UFG is very challenging
  to calculate from first principles since the problem is ``maximally
  non-perturbative'' due to the lack of a small parameter to expand
  in.  Over recent years, a large number of Quantum Monte Carlo and other
  non-perturbative calculations of the UFG have been
  accumulated. Therefore, an interesting and challenging problem will
  be to apply the IMSRG and CC methods to this system and
  compare to other methods.

\end{enumerate}



\subsubsection{Broader Impacts}

The academic environment at MSU provides numerous opportunities
for the PIs to attract bright young people to 
high-level research in nuclear physics. For example, the theory group at the NSCL is very active in mentoring undergraduate
students in the NSF sponsored Research Experience for Undergraduates
(REU) program in the summer months, with the PIs successfully supervising two REU students 
over the past three years. Many of the projects in this proposal
can be tested in prototype toy-model problems that manage to
simultaneously i) illustrate cutting-edge concepts to the student at a technical level
appropriate for an undergraduate physics major, and ii) benefit the
overall progress of the PIs' research program. 

One of the primary \emph{broader impacts} of our previous NSF award, which we will continue in the current award, is the development of an
open source library that can be used by other theorists and serve
as a educational resource for graduate students and postdocs learning about 
advanced many-body methods like CC and IMSRG.  In our recent
Lecture Notes in Physics book~\cite{lnp}, we have together with our graduate students
and other colleagues, written two long chapters on the application of
the above many-methods to neutron star studies. These chapters contain
links to our codes, which are fully open source and contain
benchmark calculations as well, making thereby our science
reproducible. They can be accessed via the GitHub
link \url{https://github.com/ManyBodyPhysics/LectureNotesPhysics/tree/master/doc/src}. As we implement the infinite matter projects described in the current proposal (e.g., inclusion of NNN forces and approximate triples excitations, calculation of spin-isospin response functions, etc.), the codes in the repository will be updated accordingly.  

The PIs are requesting support for three graduate students plus one postdoc to work on
the projects outlined above. The methods at the center of this proposal
are at the forefront of basic research in many-body physics and
low-energy nuclear physics.The many collaborations of the PIs with academic institutions world wide provides an excellent
platform for developing the communication skills of the supported
students. Global collaboration, and collaboration within the local
group, requires skills in project management in order to adequately
communicate the progress of the project and to meet
deadlines. Participation in international conferences and visits at
research centers can be expected within the time period the proposed
project. This will equip the involved participants with skills in
presentation techniques.  Proper documentation, both internally and in
peer-reviewed international journals, in writing is expected. The
project is grounded in nuclear physics but there is a very large
overlap with computational science and numerical analysis. These two
complementary aspects will form a natural part of the everyday work
and therefore add to the total competence of the
participants. Naturally, this will add to the number of possible
career paths.



