\documentclass[11pt]{article}
\usepackage{indentfirst}
\usepackage[colorlinks=True, urlcolor=blue]{hyperref}
\usepackage[dvips]{graphicx}
\usepackage{pst-plot}
\usepackage{mathrsfs}
\usepackage{epic,eepic}
\usepackage{amsfonts}
\usepackage{amsmath,amssymb}
\usepackage{bm}
\usepackage{enumitem}
\usepackage{subcaption}
\usepackage{braket}
\usepackage{tabto}
% set up paper size
\setlength{\textwidth}{17.59cm}
\setlength{\marginparsep}{0pt}
\setlength{\marginparwidth}{0pt}
\setlength{\textheight}{24.05cm}
\setlength{\headheight}{0pt}
\setlength{\headsep}{0pt}
\setlength{\oddsidemargin}{-0.04cm}
\setlength{\topmargin}{-.04cm}
\usepackage{pdfpages} 

\renewcommand{\baselinestretch}{1.05}
\DeclareMathOperator{\sgn}{sgn}

\newcommand{\dagg}[1]{\ensuremath #1^\dag}

\newcommand{\tbd}[1]{{\color{red}#1}}

\begin{document}

\title{From Quarks to Stars:  A Quantum Computing Approach to the Nuclear Many-Body Problem}

\maketitle

\begin{abstract}
  
How do we connect fundamental physics to forefront experiments?  With
new science waiting to be discovered at experimental facilities such
as FRIB, this question is a profound challenge and opportunity for
nuclear theory.  We would like to know how nuclear forces emerge from
the interactions of quarks and gluons in quantum chromodynamics, what
is the internal partonic structure of nucleons, what are the
signatures of physics beyond the Standard Model in atomic nuclei, how
do we predict nuclear structure and reactions from the microscopic
interactions of nucleons, and what are the properties of
strongly-interacting matter under extreme conditions.  There have been
many important and profound advances in recent years, with some of the
most creative and impactful developments led by investigators on this
proposal.  But there are limits to what can be done even with the most
powerful supercomputers in this soon-to-be era of exascale computing.
Unfortunately, some of the most interesting scientific questions
remain unexplored such as the real-time dynamics of hadronic and
nuclear reactions and the properties of dense nuclear and neutron
matter. For computational methods that construct quantum wave
functions, the problem is the curse of large dimensionality and the
impossibility of storing exponentially large vectors as classical
bits.  For computational methods that rely on Monte Carlo simulations,
the obstacle is the Monte Carlo sign problem, where positive and
negative contributions cancel to produce exponentially small signal-to-noise ratios.

Quantum computing has emerged as a new computational paradigm that
offers the hope of evading both of these fundamental problems.  By
allowing for arbitrary quantum superpositions of
qubits, one can store exponentially more information than with classical
bits.  Furthermore, qubits naturally evolve with unitary real-time
dynamics. Clearly there are tremendous opportunities in this new
paradigm for solving the most profound problems in nuclear physics,
but there are enormous challenges to realizing their promise.  All of the currently available digital quantum computing devices suffer from short decoherence times and significant readout errors.
Meanwhile all of the currently available
analog quantum simulators have inherent quantum characteristics that
are not directly applicable to address systems of relevance to nuclear
physics.

In order to meet these formidable challenges, we have proposed a
series of new projects that bridge the gap between quantum computing
as it exists today and the targeted territory of unsolved nuclear
physics problems.  Some of the projects focus on variational methods
and developing new algorithms for optimizing quantum wave functions.
We also explore new applications of error stabilization techniques,
machine learning, and quantum compiling to make efficient short-depth
quantum circuits. Other proposed projects focus on real-time
evolution, the calculation of observables using new algorithms, and
creative adaptations of quantum simulators to nuclear physics.  We are
confident that the scientific output of this collaboration will have a
significant and lasting impact on the landscape of quantum computing
applied to nuclear physics, and that the new algorithms and
methods we develop will have an equally
important impact on quantum computing and quantum information science.
\end{abstract}  


\end{document}
